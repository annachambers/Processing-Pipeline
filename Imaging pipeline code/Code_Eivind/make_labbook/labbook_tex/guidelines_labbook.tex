\chapter{Guidelines for Keeping Laboratory Journals}

\section*{Introduction}

\begin{itemize}
\item The keeping of laboratory journals is mandatory in all
experimental disciplines.
\item Journals should be kept in such a way that others can replicate
the experiments.
\item The main purpose of the journal is to document laboratory work
that may be published in due course.
\item Journals are the property of UiO, and must be submitted in
connection with administrative check-out or when the project is
wound up. Laboratory journals must be kept securely by the
institute or department for at least 10 years after the
experiments have been completed.
\item A secondary objective of laboratory journals is to form a basis
for patents.
\end{itemize}

\section*{General Guidelines}

\begin{itemize}
\item Supervisors/group leaders/project managers are responsible for
ensuring that all those involved in their group or project keep a
laboratory journal according to current guidelines.
\item A laboratory journal is personal and must be kept by one
person.
\item Failure to submit journals on termination of employment may
prevent administrative check-out and the issue of diplomas and
may be prosecuted. It is permissible to copy one's own
laboratory journal and take it with one to a new place of work.
\end{itemize}

\section*{Keeping of Journals}

\begin{itemize}
\item Do not use slang expressions or jargon without an explanation.
\item Never remove pages.
\item Never remove data or text from the journals. Put a line through the
paragraph, and initial it. Corrections should be written near the change.
\item All experiments should have a:
\begin{itemize}
\item Title (name)
\item Date
\item Methods
\item Results
\item Conclusion
\item There should preferably be a brief description of the objective of the experiment.
\end{itemize}

\item Label all figures and enter calculations with designations.

\item Methods must be described precisely.
\begin{itemize}
\item Include relevant: environmental factors,
equipment, material etc.
\item Is a SOP (standard operating procedure)
available?
\item If not, a complete description of the method
must be included.
\item Describe all departures from written
procedures.
\end{itemize}

\item Laboratory journals must include all relevant information and references
to supplementary information.
\begin{itemize}
\item URL
\item database files
\item PC files

\end{itemize}
\end{itemize}
